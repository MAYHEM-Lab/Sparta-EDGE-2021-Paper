In this paper, we propose a heat budget-based scheduling framework, called Sparta, aiming to prevent edge cloud CPU overheating in executing machine learning applications. Sparta's scheduler integrates three components -- data plane, decision plane, and control plane: Decision plane extrapolates the initial CPU frequency from historical benchmark data and dynamically adjusts it based on real-time data monitored by data plane, while control plane modified the CPU frequency via DVFS throughout the execution. Sparta strives to accelerate the execution of applications without sacrificing the CPU overheating protection.  

We present the design principles and implementation details of Sparta's components and operating modes that address the drawback we encounter in the testing phase. Our empirical evaluation demonstrates Sparta effectively protects CPU from overheating, putting \textbf{94.4\%} temperature samples under the threshold in Hybrid mode. In the meantime, it speeds six benchmarks' execution up to \textbf{1.04x} - \textbf{1.32x} in all three thermal environments compared to Annealing and AIMD.

As part of future work, we plan to investigate using non-uniform distributions in generating random values for exploration in Annealing mode that potentially improves the PTBT metrics. We also plan to extend the deployment of Sparta at edge cloud clusters and investigate its performance in the distributed execution of training and inference process. 
As related work, we consider recent advances in edge cloud's energy consumption and power management.~\cite{ref:computational_sprint} proposes computational sprinting which is a class of mechanisms that supplies additional power on processors for short duration to improve performance. It also introduces phase change materials onto processors to absorb additional heat primarily concerning the performance. ThriftyEdge~\cite{ref:thrifty_edge} presents a resource-efficient edge computing paradigm that consists of an offloading mechanism based on delay-aware task graph partition and a virtual machine selection method. To augment existing resources, \cite{ref:fog} manifests a dynamic fog computing framework that schedules computing tasks to Citizen Fog (CF) with the highest computational ability. Different from the above systems, Sparta focuses on preventing CPU overheating caused by ambient temperature and program execution patterns on edge cloud deployed in natural conditions. 

By offering distributed, reliable, and low-latency machine learning services, edge-based ML as a fast-growing area has a great appeal both for AI and system research community. Thus, we also consider the cutting-edge development in machine learning based on edge cloud. \cite{ref:wireless} explores the building blocks and principles of wireless intelligence at edge networks concerning latency reduction, reliability guarantees, scalability enhancement, and privacy constraints. \cite{ref:survey} provides a comprehensive survey of techniques in the scope of machine learning system at the network edge, including distributed training and inference, real-time video analytics and speech recognition, autonomous vehicles and smart cities, etc. \cite{ref:stargazer} presents an approach to estimate the performance of ML application on edge cloud and to load appropriate computing resources for an edge-based application. The above work provide guiding principles and examples for Sparta and serve as one of the key motivations for our work.
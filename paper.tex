% This is samplepaper.tex, a sample chapter demonstrating the
% LLNCS macro package for Springer Computer Science proceedings;
% Version 2.20 of 2017/10/04
%
\documentclass[runningheads]{llncs}
%
\usepackage{graphicx}
\usepackage{gensymb}
\usepackage{array}
\usepackage{color, colortbl}
\usepackage[labelfont=bf]{caption}

\newcommand{\ignore}[1]{}


% \usepackage{makecell}
% Used for displaying a sample figure. If possible, figure files should
% be included in EPS format.
%
% If you use the hyperref package, please uncomment the following line
% to display URLs in blue roman font according to Springer's eBook style:
% \renewcommand\UrlFont{\color{blue}\rmfamily}

\usepackage{xcolor}
\usepackage{caption} %needed to make captions on figure* centered
\graphicspath{{./figures/}}
\DeclareGraphicsExtensions{.pdf,.png}
\usepackage[bookmarks=false]{hyperref}
\usepackage[linesnumbered, ruled, vlined]{algorithm2e}

\definecolor{Gray}{gray}{0.9}
\definecolor{LightCyan}{rgb}{0.88,1,1}

% correct bad hyphenation here
\hyphenation{Thrifty-Edge}


\begin{document}
%

\title{Sparta: A Heat-Budget-based Scheduling Framework on IoT Edge Systems}

\titlerunning{Sparta}
%
%\titlerunning{Abbreviated paper title}
% If the paper title is too long for the running head, you can set
% an abbreviated paper title here
%


\author{Michael Zhang\inst{1} \and
Chandra Krintz\inst{1} \and
Rich Wolski\inst{1}}

\authorrunning{Zhang, Krintz, Wolski}

\institute{Department of Computer Science\\ University of California, Santa Barbara, CA 93106, USA \\
\email{\{lebo, ckrintz, rich\}@cs.ucsb.edu}}
%
\maketitle              % typeset the header of the contribution
%


\begin{abstract}
Co-location of processing infrastructure and IoT devices at the edge is used to reduce response latency and long-haul network use for IoT applications. As a result, edge clouds for many applications (e.g. agriculture, ecology, and smart city deployments) must operate in remote, unattended, and environmentally harsh settings, introducing new challenges. One key challenge is heat exposure, which can degrade the performance, reliability, and longevity of electronics. For edge clouds, these problems are exacerbated because they increasingly perform complex workloads, such as machine learning, to affect data-driven actuation and control of devices and systems in the environment.

\vspace{0.08in}
The goal of our work is to protect edge clouds from overheating. To enable this, we develop a heat-budget-based scheduling system, called Sparta, which leverages dynamic voltage and frequency scaling (DVFS) to adaptively control CPU temperature. Sparta takes machine learning applications, datasets, and a temperature threshold as input. It sets the initial frequency of the CPU based on historical data and then dynamically updates it, according to the applications' execution profile and ambient temperature, to safeguard edge devices. We find that for a suite of machine learning applications and deployment temperatures, Sparta is able to maintain CPU temperature below the threshold 94\% of the time while facilitating improvements in execution time by 1.04x - 1.32x over competitive approaches.


\keywords{Edge Computing, Heat Budget, Scheduling System, IoT}
\end{abstract}
%
%
%


\section{Introduction}
\label{sec:intro}
The Internet of Things (IoT) is a rapidly emerging
set of technologies in which ordinary objects are equipped with digital
intelligence -- the ability to sense, analyze, and control their
environment automatically. By linking our physical and digital worlds, IoT
has the potential to to enhance
situational awareness and effective decision-making by humans, to detect,
diagnose, and remediate problems without human intervention, to assist with
personal and homeland security, to optimize manufacturing and business
processes, and to automate operations throughout our economy.

To realize this impact, IoT deployments are embedded in the world around 
us -- within buildings, cars, roads, homes, industrial machinery, 
and waterways, and distributed across farms, wild open spaces, cities, and oceans.
Moreover, they increasingly leverage
recent advances in data analytics, machine learning (ML), and automation in-situ -- at
the edge of the network (``near'' data/control sources).
This move to the edge is the result of an increase in the velocity and volume of
data and high response latencies imposed by the
long-haul, intermittently available networks that connect the edge and cloud.
Co-location of processing infrastructure and IoT devices
significantly reduces the latency between data acquisition and device
actuation, enables extension of device capability via local offloading,
and alleviates the cost, power consumption, and congestion of network use
versus the centralized, cloud-direct
model~\cite{edge,bonomi2012fog,cloudlets,cloudlets2012satya,verbelen2012cloudlets}.

Edge processing however introduces new challenges for IoT deployments.
The operational settings in which these systems (i.e. edge clouds) are deployed
can be harsh, hard or costly to access, and exposed to harmful
environmental elements (heat, moisture, dust, animals, other objects, humans, weather, etc.).
For example, we currently support an IoT deployment for image processing and 
deep learning for the automatic, real time identification of animals
using camera traps deployed across UCSB Sedgwick Reserve, an
ecology and wildlife educational and research reserve in California~\cite{ref:sedgwick}. 
The reserve is 6,000 acres that comprises critical wildlife habitats, 
two watersheds at the foot of Figueroa Mountain, and a 300 acre farm easement. 
Our edge clouds fuse and analyze images
from within out-buildings on the property. Sedgwick yearly
temperatures range between 30$^{\circ}$ and 116$^{\circ}$ Fahrenheit; within
enclosures and out-buildings, our cloud systems are subjected 
to much higher temperatures.

Heat can degrade the performance and reliability of devices and negatively
impact their longevity (requiring more human intervention and frequent replacement).
Electronics are particularly sensitive to high temperatures and extended exposure
can cause devices to break down, degrade in functionality,
and fail prematurely -- 
even they are protected using operational safeguards 
such as throttling and automatic shutdown~\cite{ref:overheating}. 
Figure\ref{fig:time_series} shows a time series of CPU temperature for one of
our edge clouds deployed at Sedgwick between February and June. 

\begin{figure}
\includegraphics[width=\textwidth]{figures/time_series.png}
\caption{The time series of CPU temperature in the edge cloud deployed at Sedgwick Natural Reserve from Feb. 28th, 2018 to Jun. 3rd, 2020. The x-axis is the epoch time and the y-axis is the CPU temperature in Fahrenheit. } \label{fig:time_series}
\end{figure}

In this paper, we investigate a new scheduling system to reduce 
overheating of IoT edge clouds automatically. Our system, called \textbf{Sparta},
attempts to maintain a threshold CPU temperature (as ambient temperature
changes) by adjusting processor frequency, while maximizing application performance.
We use Sparta to study the relationship among CPU frequency, temperature, power dissipation, and execution behavior. Moreover, we consider IoT workloads that employ a wide range of
machine learning algorithms, including image recognition, natural language processing, decision forest and time series prediction. 

%Sparta takes an application, corresponding dataset and a configurable temperature threshold as inputs and automatically schedules the workload based on the online sampling temperature data. We use the system to perform a 
%range of machine learning algorithms, including image recognition, natural language processing, decision forest and time series prediction. This benchmark suite represents different execution patterns in machine learning applications that we utilize in the real-world IoT settings. 

We consider three modes for the Sparta scheduler: \textbf{Annealing}, \textbf{AIMD}, and \textbf{Hybrid}.  Annealing employs an epsilon-greedy strategy to extrapolate 
an appropriate CPU frequency in real time. 
AIMD uses the linear growth of CPU frequency when temperature is under threshold and 
exponential reduction when it detects temperature anomalies to determine its 
CPU frequency.  With Hybrid, we combine the best features of the two modes to overcome their drawbacks. Our results show that Sparta in Hybrid mode speeds up the execution of our applications by \textbf{1.16x} and \textbf{1.14x} on average in three thermal environments compared to Annealing and AIMD. Moreover, Sparta in Hybrid mode maintains CPU
temperature below threshold \textbf{94.4\%} of the time (as measured via temperature
sampling), on average across all benchmarks. 

In summary, with this paper, we make the following contributions:
\begin{itemize}
    \item We investigate the relationship between CPU frequency and sampling temperature to precisely model and manage processor power dissipation during execution;
    \vspace{1mm}
    \item We design and implement a heat-budget-based scheduling framework that protects edge systems from overheating and potential damage;
    \vspace{1mm}
    \item We empirically evaluate the efficacy of using Sparta to control CPU temperature and accelerate machine learning applications on six real-world benchmarks in three thermal deployment environments. 
\end{itemize}
In the following sections, we present the design and implementation of Sparta (Section~\ref{sec:Sparta}). We then describe our experimental methodology and empirical evaluation of the system using multiple machine learning applications in thermal environments (Section~\ref{sec:eval}). In Section~\ref{sec:relate_work},  we discuss related work. Finally, we present our conclusions and future work plans.

\ignore{
Cloud computing enables the delivery of numerous computing services, including processing power, data storage, data analytics, networking and software, over the Internet. The demand on scalable and agile services has triggered an ecosystem of cloud computing, in which users can optimize cloud systems by integrating virtualized resources and obtain cost advantage compared to in-house IT infrastructure. In addition, the autoscaling of elastic services automatically allocate computational resources when the production workloads are unpredictable, allowing the system to handle the variable traffic spikes better.

To address the security concern and networking latency, cloud computing has evolved into a stratified system that contains IoT clusters, edge cloud and data centers. In this paradigm, edge cloud serves as the middle layer between IoT devices and data centers that brings the computational power and data storage near the deployment sites in hope of improving round-trip time and bandwidth usage. Thus, in the scenarios like real-time data streaming from sensors or users, edge cloud plays a critical role in timely execution and content delivery. 

Particularly, with the arrival of powerful edge cloud devices, computing tasks that requires significant computational power can now be executed on edge devices. Such capability brings more complex machine learning models that analyzes large amounts of dataset closer to data sources and reduces networking latency. This movement has given rise to "edge-based" machine learning that deploys advanced algorithms such as convolutional neural networks (CNNs) at the edge of the network.
}


\section{Sparta}
\label{sec:Sparta}
% Before executing the task, Sparta extrapolates the CPU frequency from the historical data and current CPU temperature, and then tunes the CPU frequency via Dynamic voltage and frequency scaling (DVFS).

\section{Evaluation}
\label{sec:eval}
In this section, we empirically evaluate Sparta's performance in a series of experiments on six benchmarks, ranging from image recognition, natural language processing to random forest and time series prediction. We implemented applications based on Tensorflow and executed through Sparta's actuator interface.  In each experiment, Sparta takes inputs of task program, workload dataset and threshold temperature. There are two goals of scheduling by Sparta: the first is to limit the CPU temperature under the threshold; secondly, Sparta accelerates the task to the fullest extent without overheating the edge cloud devices.

\subsection{Machine Learning Benchmarks}

To comprehensively evaluate the efficacy and efficiency of Sparta, we implemented 6 machine learning benchmarks, which consist of four categories: image recognition, natural language processing, ensemble learning and time series analysis. We aim to test Sparta on a variety of machine learning applications that represent different execution patterns.

\subsubsection{WTB\_Train}

is the first image recognition application that we use as a benchmark trains a convolutional neural network (CNN)~\cite{ref:cnn} on the top of ResNet50~\cite{ref:resnet}. The training dataset contains animal images from a wildlife monitoring system called "Where's The Bear" (WTB)~\cite{ref:wtb}. "Where's The Bear" is an end-to-end distributed data acquisition and analytical system that automatically analyzes camera trap images collected by cameras sited at the Sedgwick Natural Reserve~\cite{ref:sedgwick} in Santa Barbara County, California. In total, there are five classes that we consider: Bear, Coyote, Deer, Bird and Empty, by which we label images for training tasks. We also up-sampled minority classes using the Keras Image Data Generator~\cite{ref:datagen}, since the class size is unbalanced due to the frequency of animal occurrences. Doing so ensures that the classification model is not biased. We resized every image in the dataset to $1920 \times 1080$, and for each class, the dataset contains 251 images used to train the CNN model. Once the training is complete, the application stores this model in hdf5 format in object storage. 

The WTB\_Train application has a cold start at the beginning of the execution, since it loads a neural network model and a large dataset. Once it completes loading, the entire training process has relatively consistent CPU usage and temperature. 

\subsubsection{WTB\_Inf}

Based on WTB\_Train, we inference the type of wildlife in camera trap pictures by the second application. It loads the trained model from WTB\_Train and, for each picture, it assigns probabilities to five classes we consider in the training dataset by Softmax function. In terms of the execution pattern, WTB\_Inf runs in short bursts compared to WTB\_Train. Therefore, the CPU usage and temperature fluctuate dramatically throughout the execution of this benchmark.

\subsubsection{MNIST}

 is a dataset containing grayscale pictures of handwritten digits, in which it has 60,000 examples as training set and 10,000 examples as testing set. Based on the dataset, we train a 2-layer convolutional neural network~\cite{ref:MNIST} and test its accuracy in the third application. In contrast to WTB benchmarks, the size of pictures is smaller ($28 \times 28$) and the model is simplified in MNIST. We aim to evaluate Sparta on an application that includes both training and inference process. 

\subsubsection{BiLSTM} is a sentiment analysis application based on a dataset of the Internet Movie Database (IMDB) movie reviews. It consists of 25,000 sequences each for training and testing. The model is constructed as a bidirectional LSTM with a classification layer using sigmoid activation function. We train the model by the training dataset and validate by testing dataset in BiLSTM application. Since it has a large dataset and a complex model, the execution pattern is long-running and consistent in CPU usage and temperature.


\subsubsection{Decision\_Forest} is an implementation of deep neural decision forests~\cite{ref:decision_forest} that classifies high-earning individuals from the pool. The benchmark leverages the United States Census Income Dataset~\cite{ref:uci} that has 48,843 instances with 14 features, including age, education, occupation, etc. The dataset is split up that the training part has 32,561 instances and the testing part has 16,282 instances. The application has three phases: it firstly processes the dataset by encoding input features. Then, it trains a deep neural decision tree model. Based on that, the application trains a neural decision forest model consists of a set of neural decision trees. Therefore, the usage and temperature of CPU increasingly grows throughout the process.


\subsubsection{Time\_Series} is a time series prediction application built on the climate data recorded by the Max Planck Institute for Biogeochemistry~\cite{ref:jena}. The dataset has 14 features such as temperature, pressure, humidity, etc. and the sampling frequency is 10 minutes. The time frame of the dataset ranges from Jan. 10th, 2009 to Dec. 31st, 2016. The application uses 300,693 rows to train a single-layer LSTM model, by which we can predict temperature in next 72 timestamps (12 hours) given the temperature in the past 720 timestamps (120 hours). We intend to evaluate Sparta on an application with light model and large dataset.

% Execution pattern table

\subsection{Experimental Setup}

The edge cloud device we set up for experiments is Intel NUC~\cite{ref:nuc} (6i7KYK) with two Intel Core i7-6770HQ 4-core processors (6M Cache, 2.60 GHz) and 32GB of DDR4-2133+ RAM connected via two channels.  

To simulate the natural temperature in Sedgwick natural reserve, we create three thermal environments in an isolated cooler that represent cold, neutral and hot ambient temperature. In the cold scenario, the ambient temperature is 2.6\degree C and the CPU of NUC runs under 40\degree C in idle status. In the neutral scenario, the CPU of NUC starts at 51\degree C under the ambient temperature of 23.9\degree C. The hot scenario increases the ambient and CPU temperature to 43.8\degree C and 68\degree C respectively.

\begin{figure}
\includegraphics[width=\textwidth]{figures/thermal.jpg}
\caption{Three thermal environments in the experiment} \label{cold}
\end{figure}

\subsection{}
 

\section{Related Work}
\label{sec:relate_work}
As related work, we consider recent advances in edge cloud's energy consumption and power management.~\cite{ref:computational_sprint} proposes computational sprinting which is a class of mechanisms that supplies additional power on processors for short duration to improve performance. It also introduces phase change materials onto processors to absorb additional heat primarily concerning the performance. ThriftyEdge~\cite{ref:thrifty_edge} presents a resource-efficient edge computing paradigm that consists of an offloading mechanism based on delay-aware task graph partition and a virtual machine selection method. To augment existing resources, \cite{ref:fog} manifests a dynamic fog computing framework that schedules computing tasks to Citizen Fog (CF) with the highest computational ability. Different from the above systems, Sparta focuses on preventing CPU overheating caused by ambient temperature and program execution pattern on edge cloud deployed in natural conditions. 

By offering distributed, reliable and low-latency machine learning services, edge-based ML as a fast-growing area has a great appeal both for AI and system research community. Thus, we also consider the cutting-edge development in machine learning based on edge cloud. \cite{ref:wireless} explores the building blocks and principles of wireless intelligence at edge network concerning latency reduction, reliability guarantees, scalability enhancement and privacy constraints. \cite{ref:survey} provides a comprehensive survey of techniques in scope of machine learning system at the network edge, including distributed training and inference, real-time video analytics and speech recognition, autonomous vehicles and smart cities, etc. \cite{ref:stargazer} presents an approach to estimate the performance of ML application on edge cloud and to load appropriate computing resources for an edge-based application. The above work provide guiding principles and examples for Sparta and serve as one of the key motivations for our work.

\section{Conclusion}
\label{sec:conclusion}
In this paper, we propose a heat budget-based scheduling framework, called Sparta, aiming to prevent edge cloud CPU overheating in executing machine learning applications. Sparta's scheduler integrates three components -- data plane, decision plane, and control plane: Decision plane extrapolates the initial CPU frequency from historical benchmark data and dynamically adjusts it based on real-time data monitored by data plane, while control plane modified the CPU frequency via DVFS throughout the execution. Sparta strives to accelerate the execution of applications without sacrificing the CPU overheating protection.  

We present the design principles and implementation details of Sparta's components and operating modes that address the drawback we encounter in the testing phase. Our empirical evaluation demonstrates Sparta effectively protects CPU from overheating, putting \textbf{94.4\%} temperature samples under the threshold in Hybrid mode. In the meantime, it speeds six benchmarks' execution up to \textbf{1.04x} - \textbf{1.32x} in all three thermal environments compared to Annealing and AIMD.

As part of future work, we plan to investigate using non-uniform distributions in generating random values for exploration in Annealing mode that potentially improves the PTBT metrics. We also plan to extend the deployment of Sparta at edge cloud clusters and investigate its performance in the distributed execution of training and inference process. 


%
% ---- Bibliography ----
%
% BibTeX users should specify bibliography style 'splncs04'.
% References will then be sorted and formatted in the correct style.
%

\bibliographystyle{splncs04}
% \bibliography{ref}

\begin{thebibliography}{8}

\bibitem{edge}
N. Hassan, S. Gillani, E. Ahmed, I. Yaqoob and M. Imran, "The Role of Edge Computing in Internet of Things," in IEEE Communications Magazine, vol. 56, no. 11, pp. 110-115, November 2018, doi: 10.1109/MCOM.2018.1700906.

\bibitem{bonomi2012fog}
Flavio Bonomi, Rodolfo Milito, Jiang Zhu, and Sateesh Addepalli. 2012. Fog computing and its role in the internet of things. In Proceedings of the first edition of the MCC workshop on Mobile cloud computing (MCC '12). Association for Computing Machinery, New York, NY, USA, 13–16. DOI:https://doi.org/10.1145/2342509.2342513

% \bibitem{cloudlets}
% Cloudlet-based Mobile Computing
% \url {http://elijah.cs.cmu.edu/} Last accessed 30 Apr 2021

\bibitem{cloudlets2012satya}
S. Simanta, G. A. Lewis, E. Morris, K. Ha and M. Satyanarayanan, "A Reference Architecture for Mobile Code Offload in Hostile Environments," 2012 Joint Working IEEE/IFIP Conference on Software Architecture and European Conference on Software Architecture, 2012, pp. 282-286, doi: 10.1109/WICSA-ECSA.212.46.

\bibitem{verbelen2012cloudlets}
Tim Verbelen, Pieter Simoens, Filip De Turck, and Bart Dhoedt. 2012. Cloudlets: bringing the cloud to the mobile user. In Proceedings of the third ACM workshop on Mobile cloud computing and services (MCS '12). Association for Computing Machinery, New York, NY, USA, 29–36. DOI:https://doi.org/10.1145/2307849.2307858

\bibitem{ref:sedgwick}
Sedgwick Natural Reserve Homepage \url{https://sedgwick.nrs.ucsb.edu} Last accessed 30 Apr 2021

\bibitem{ref:overheating}
\url{https://www.intel.com/content/www/us/en/support/articles/000005597/processors.html} Last accessed 30 Apr 2021 


\bibitem{ref:Liu2007dvfs}
Liu, Yongpan, Huazhong Yang, Robert P. Dick, Hui Wang, and Li Shang. "Thermal vs energy optimization for dvfs-enabled processors in embedded systems." In 8th International Symposium on Quality Electronic Design (ISQED'07), pp. 204-209. IEEE, 2007.


\bibitem{ref:Wang2010dvfs}
Wang, Lizhe, Gregor Von Laszewski, Jay Dayal, and Fugang Wang. "Towards energy aware scheduling for precedence constrained parallel tasks in a cluster with DVFS." In 2010 10th IEEE/ACM International Conference on Cluster, Cloud and Grid Computing, pp. 368-377. IEEE, 2010.

% \bibitem{ref:Deng2012dvfs}
% Deng, Qingyuan, David Meisner, Abhishek Bhattacharjee, Thomas F. Wenisch, and Ricardo Bianchini. "CoScale: Coordinating CPU and memory system DVFS in server systems." In 2012 45th annual IEEE/ACM international symposium on microarchitecture, pp. 143-154. IEEE, 2012.


\bibitem{ref:Wu2013dvfs}
Chia-Ming Wu, Ruay-Shiung Chang, Hsin-Yu Chan,
A green energy-efficient scheduling algorithm using the DVFS technique for cloud datacenters,
Future Generation Computer Systems,
Volume 37,
2014,
Pages 141-147,
ISSN 0167-739X,
https://doi.org/10.1016/j.future.2013.06.009.

\bibitem{ref:sensors}
\url{https://github.com/lm-sensors/lm-sensors} Last accessed 30 Apr 2021 

\bibitem{ref:tensorflow}
\url{https://www.tensorflow.org/} Last accessed 30 Apr 2021 


\bibitem{ref:cnn}
Yann LeCun and Yoshua Bengio. Convolutional networks for images, speech, and time series. The handbook of brain theory and neural networks. MIT Press, Cambridge, MA, USA, pp. 255–258, 1998.

\bibitem{ref:resnet}
He, Kaiming, Xiangyu Zhang, Shaoqing Ren, and Jian Sun. "Deep residual learning for image recognition." In Proceedings of the IEEE conference on computer vision and pattern recognition, pp. 770-778. 2016.

\bibitem{ref:wtb}
A. R. Elias, N. Golubovic, C. Krintz and R. Wolski, "Where's the Bear? - Automating Wildlife Image Processing Using IoT and Edge Cloud Systems," 2017 IEEE/ACM Second International Conference on Internet-of-Things Design and Implementation (IoTDI), pp. 247-258, 2017.

\bibitem{ref:datagen}
Keras Image Data Generator \url{https://keras.io/preprocessing/image/\#imagedatagenerator-class} Last accessed 30 Apr 2021



\bibitem{ref:MNIST}
Y. LeCun, L. Bottou, Y. Bengio and P. Haffner: Gradient-Based Learning Applied to Document Recognition, Proceedings of the IEEE, 86(11):2278-2324, November 1998

\bibitem{ref:decision_forest}
P. Kontschieder, M. Fiterau, A. Criminisi and S. R. Bulò, "Deep Neural Decision Forests," 2015 IEEE International Conference on Computer Vision (ICCV), 2015, pp. 1467-1475, doi: 10.1109/ICCV.2015.172.

\bibitem{ref:uci}
\url{https://archive.ics.uci.edu/ml/datasets/census+income} Last accessed 30 Apr 2021

\bibitem{ref:jena}
\url{https://www.bgc-jena.mpg.de/wetter/} Last accessed 30 Apr 2021 

\bibitem{ref:nuc}
\url{https://www.intel.com/content/www/us/en/products/boards-kits/nuc.html} Last accessed 30 Apr 2021 









\bibitem{ref:computational_sprint}
Seyed Majid Zahedi, Songchun Fan, Matthew Faw, Elijah Cole, and Benjamin C. Lee. 2017. Computational Sprinting: Architecture, Dynamics, and Strategies. ACM Trans. Comput. Syst. 34, 4, Article 12 (January 2017), 26 pages. DOI:https://doi.org/10.1145/3014428


\bibitem{ref:thrifty_edge}
X. Chen, Q. Shi, L. Yang and J. Xu, "ThriftyEdge: Resource-Efficient Edge Computing for Intelligent IoT Applications," in IEEE Network, vol. 32, no. 1, pp. 61-65, Jan.-Feb. 2018, doi: 10.1109/MNET.2018.1700145.


\bibitem{ref:fog}
Md Razon Hossain, Md Whaiduzzaman, Alistair Barros, Shelia Rahman Tuly, Md. Julkar Nayeen Mahi, Shanto Roy, Colin Fidge, Rajkumar Buyya,
A scheduling-based dynamic fog computing framework for augmenting resource utilization,
Simulation Modelling Practice and Theory,
Volume 111, 2021, 102336, ISSN 1569-190X,
https://doi.org/10.1016/j.simpat.2021.102336.


\bibitem{ref:wireless}
Park, Jihong \& Samarakoon, Sumudu \& Bennis, Mehdi \& Debbah, mérouane. (2019). Wireless Network Intelligence at the Edge. Proceedings of the IEEE. 107. 10.1109/JPROC.2019.2941458. 


\bibitem{ref:survey}
Murshed, M. G. Sarwar \& Murphy, Christopher \& Hou, Daqing \& Khan, Nazar \& Ananthanarayanan, Ganesh \& Hussain, Faraz. (2019). Machine Learning at the Network Edge: A Survey. 


\bibitem{ref:stargazer}
B. D. Cruz, A. K. Paul, Z. Song and E. Tilevich, "Stargazer: A Deep Learning Approach for Estimating the Performance of Edge- Based Clustering Applications," 2020 IEEE International Conference on Smart Data Services (SMDS), 2020, pp. 9-17, doi: 10.1109/SMDS49396.2020.00009.



\end{thebibliography}




\end{document}
